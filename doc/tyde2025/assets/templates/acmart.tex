\documentclass[sigplan,screen]{acmart}

% TODO:
% Validate packages against the ACM TAPS accepted package list
% See: https://authors.acm.org/proceedings/production-information/accepted-latex-packages

\undef\Bbbk%
% lhs2TeX:
%include polycode.fmt

${ if(lang) }
\ifLuaTeX
\usepackage[bidi=basic${ for(babeloptions) },${ babeloptions }${ endfor }]{babel}
\else
\usepackage[bidi=default${ for(babeloptions) },${ babeloptions }${ endfor }]{babel}
\fi
${ if(babel-lang) }
${ if(mainfont) }
\ifPDFTeX
\else
\babelfont{rm}[${ for(mainfontoptions) }${ mainfontoptions }${ sep },${ endfor }${ if(mainfontfallback) },RawFeature={fallback=mainfontfallback}${ endif }]{${ mainfont }}
\fi
${ endif }
${ endif }
${ for(babelfonts/pairs) }
\babelfont[${ babelfonts.key }]{rm}{${ babelfonts.value }}
$endfor$
% get rid of language-specific shorthands (see #6817):
\let\LanguageShortHands\languageshorthands
\def\languageshorthands#1{}
${ if(selnolig-langs) }
\ifLuaTeX
  \usepackage[${ for(selnolig-langs) }${ it }${ sep },${ endfor }]{selnolig} % disable illegal ligatures
\fi
${ endif }
${ endif }

%% prevent overfull lines
\setlength{\emergencystretch}{3em} % prevent overfull lines

%% tight lists
\providecommand{\tightlist}{%
  \setlength{\itemsep}{0pt}\setlength{\parskip}{0pt}}

% Preamble:
${ for(header-includes) }
${ header-includes }
${ endfor }

%% Rights management information.  This information is sent to you
%% when you complete the rights form.  These commands have SAMPLE
%% values in them; it is your responsibility as an author to replace
%% the commands and values with those provided to you when you
%% complete the rights form.
${ if(copyright) }
\setcopyright{${ copyright }}
${ else }
\setcopyright{none}
${ endif }
${ if(copyrightyear) }
\copyrightyear{${ copyrightyear }}
${ endif }
${ if(acmYear) }
\acmYear{${ acmYear }}
${ endif }
${ if(acmDOI) }
\acmDOI{${ acmDOI }}
${ else }
\acmDOI{XXXXXXX.XXXXXXX}
${ endif }

%% These commands are for a PROCEEDINGS abstract or paper.
% \acmConference[Conference acronym 'XX]{Make sure to enter the correct
%   conference title from your rights confirmation email}{June 03--05,
%   2018}{Woodstock, NY}
%%
%%  Uncomment \acmBooktitle if the title of the proceedings is different
%%  from ``Proceedings of ...''!
%%
%%\acmBooktitle{Woodstock '18: ACM Symposium on Neural Gaze Detection,
%%  June 03--05, 2018, Woodstock, NY}
% \acmISBN{978-1-4503-XXXX-X/2018/06}

%%
%% Submission ID.
%% Use this when submitting an article to a sponsored event. You'll
%% receive a unique submission ID from the organizers
%% of the event, and this ID should be used as the parameter to this command.
%%\acmSubmissionID{123-A56-BU3}

%%
%% For managing citations, it is recommended to use bibliography
%% files in BibTeX format.
%%
%% You can then either use BibTeX with the ACM-Reference-Format style,
%% or BibLaTeX with the acmnumeric or acmauthoryear sytles, that include
%% support for advanced citation of software artefact from the
%% biblatex-software package, also separately available on CTAN.
%%
%% Look at the sample-*-biblatex.tex files for templates showcasing
%% the biblatex styles.
%%

%%
%% The majority of ACM publications use numbered citations and
%% references.  The command \citestyle{authoryear} switches to the
%% "author year" style.
%%
%% If you are preparing content for an event
%% sponsored by ACM SIGGRAPH, you must use the "author year" style of
%% citations and references.
%% Uncommenting
%% the next command will enable that style.
%%\citestyle{acmauthoryear}

%%
%% end of the preamble, start of the body of the document source.
\begin{document}

%%
%% The "title" command has an optional parameter,
%% allowing the author to define a "short title" to be used in page headers.
\title{${ title }}

%%
%% The "author" command and its associated commands are used to define
%% the authors and their affiliations.
%% Of note is the shared affiliation of the first two authors, and the
%% "authornote" and "authornotemark" commands
%% used to denote shared contribution to the research.
${ for(author) }
\author{${ it.name }}
${ if(it.note) }
\authornote{${ it.note }}
${ endif }
${ if(it.email) }
\email{${ it.email }}
${ endif }
${ if(it.orcid) }
\orcid{${ it.orcid }}
${ endif }
${ if(it.affiliation) }
\affiliation{
  ${ if(it.institution) }
  \institution{${ it.institution }}
  ${ endif }
  ${ if(it.city) }
  \city{${ it.city }}
  ${ endif }
  ${ if(it.state) }
  \state{${ it.state }}
  ${ endif }
  ${ if(it.country) }
  \country{${ it.country }}
  ${ endif }
}
${ endif }
${ endfor }

%% By default, the full list of authors will be used in the page
%% headers. Often, this list is too long, and will overlap
%% other information printed in the page headers. This command allows
%% the author to define a more concise list
%% of authors' names for this purpose.
${ if(shortauthors) }
\renewcommand{\shortauthors}{${ shortauthors }}
${ endif }

%% The abstract is a short summary of the work to be presented in the
%% article.
${ if(abstract) }
\begin{abstract}
  ${ abstract }
\end{abstract}
${ endif }

%%
%% The code below is generated by the tool at http://dl.acm.org/ccs.cfm.
%% Please copy and paste the code instead of the example below.
%%
\begin{CCSXML}
<ccs2012>
 <concept>
  <concept_id>00000000.0000000.0000000</concept_id>
  <concept_desc>Do Not Use This Code, Generate the Correct Terms for Your Paper</concept_desc>
  <concept_significance>500</concept_significance>
 </concept>
 <concept>
  <concept_id>00000000.00000000.00000000</concept_id>
  <concept_desc>Do Not Use This Code, Generate the Correct Terms for Your Paper</concept_desc>
  <concept_significance>300</concept_significance>
 </concept>
 <concept>
  <concept_id>00000000.00000000.00000000</concept_id>
  <concept_desc>Do Not Use This Code, Generate the Correct Terms for Your Paper</concept_desc>
  <concept_significance>100</concept_significance>
 </concept>
 <concept>
  <concept_id>00000000.00000000.00000000</concept_id>
  <concept_desc>Do Not Use This Code, Generate the Correct Terms for Your Paper</concept_desc>
  <concept_significance>100</concept_significance>
 </concept>
</ccs2012>
\end{CCSXML}

\ccsdesc[500]{Do Not Use This Code~Generate the Correct Terms for Your Paper}
\ccsdesc[300]{Do Not Use This Code~Generate the Correct Terms for Your Paper}
\ccsdesc{Do Not Use This Code~Generate the Correct Terms for Your Paper}
\ccsdesc[100]{Do Not Use This Code~Generate the Correct Terms for Your Paper}

%% Keywords. The author(s) should pick words that accurately describe
%% the work being presented. Separate the keywords with commas.
\keywords{Do, Not, Us, This, Code, Put, the, Correct, Terms, for, Your, Paper}

%% A "teaser" image appears between the author and affiliation
%% information and the body of the document, and typically spans the
%% page.
% \begin{teaserfigure}
%   \includegraphics[width=\textwidth]{sampleteaser}
%   \caption{Seattle Mariners at Spring Training, 2010.}
%   \Description{Enjoying the baseball game from the third-base
%   seats. Ichiro Suzuki preparing to bat.}
%   \label{fig:teaser}
% \end{teaserfigure}

\received{20 February 2007}
\received[revised]{12 March 2009}
\received[accepted]{5 June 2009}

%% This command processes the author and affiliation and title
%% information and builds the first part of the formatted document.
\maketitle

${ body }

%% The next two lines define the bibliography style to be used, and
%% the bibliography file.
${ if(bibliography) }
\bibliographystyle{ACM-Reference-Format}
\bibliography{${ bibliography }}
${ endif }

\end{document}
\endinput
